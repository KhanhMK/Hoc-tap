\documentclass{article}
\usepackage{graphicx} % Required for inserting images
\usepackage{amsmath,amsxtra,amssymb,latexsym, amscd,amsthm,mathrsfs}
\usepackage[utf8]{vietnam}
\usepackage[cmyk]{xcolor}
\usepackage{times}
\usepackage{enumerate}
\title{đề thi toán 12}
\author{Nguyễn Quốc Khánh }
\date{July 2023}
\usepackage[colorlinks,hyperindex,plainpages=false,unicode]{hyperref}
\newcommand{\tuyen}[1]{\left[\begin{array}{1}#1\end{array}\right.}
\begin{document}
\fontsize{13pt}{16pt}\selectfont
 \tableofcontents 
\begin{abstract}
    Bị cáo Trần Thị Mai Xa, 35 tuổi, Giám đốc Công ty Masterlife, là một trong 23 lãnh đạo doanh nghiệp bị VKSND Tối cao truy tố về tội Đưa hối lộ, trong vụ án "chuyến bay giải cứu
\end{abstract}
\begin{flushright}
    aaaaaaaaa
\end{flushright}
\section{su}
\begin{enumerate}[a).]
    \item tư liệu số 1
    \item su su
\end{enumerate}
\textcolor{red}{su}
\title{\bfseries \textcolor{blue}{TỪ BÀI TOÁN SANGAKU CỦA NHẬT BẢN ĐẾN CÁC ĐỀ THI OLYMPYA}}
 Bị cáo Trần Thị Mai Xa, 35 tuổi, Giám đốc Công ty Masterlife, là một trong 23 lãnh đạo doanh nghiệp bị VKSND Tối cao truy tố về tội Đưa hối lộ, trong vụ án "chuyến bay giải cứu". Cáo trạng xác định, để Masterlife được bay 18 chuyến, bà Xa chi hơn 8 tỷ đồng cho 8 quan chức bộ ngành \par trong đó bị cáo Vũ Anh Tuấn và Vũ Sỹ Cường, hai cựu cán bộ Cục Xuất Nhập cảnh, nhận khoảng 2,6 tỷ đồng; cựu Thứ trưởng Ngoại giao Tô Anh Dũng nhận 30.000 USD; cựu cục trưởng Lãnh sự Nguyễn Thị Hương Lan nhận 55.000 USD.  
 
 \vspace*{10cm}
\noindent\begin{tabular}{cc}
\parbox{.4\textwidth}{\centering
SỞ GD VÀ ĐT HỒ CHÍ MINH \\
ĐỀ CHÍNH THỨC \\

\vspace*{32pt}
} &\parbox{.6\textwidth}{\centering
ĐỀ KIỂM TRA CHẤT LƯỢNG HỌC KỲ I \\
 NĂM HỌC 2023-2024 \\
Môn: Toán Lớp 12 \\
Thời gian: 90 phút 
} \\

\end{tabular}
\newpage
suuu
\begin{flushright}
    hé  lô  các  bạn trẻ hêhehehehehhehehe
\end{flushright}
\begin{minipage}{.3\textwidth}
 SỞ GIÁO DỤC ĐÀO TẠO \\ HUYỆN CẦN GIỜ
\end{minipage}
$\begin{aligned}
      A &  =B \\
      C & =D \\
           & =E \\
     & =FGDFDF
\end{aligned}$
\begin{equation}
    \left ( x^2-1 \right). ^2
\end{equation}
$\lim\limits_{2\to +\infty}f(x)=3$ \\

$x^2-3x+1=0$
    
Cho tứ diện đều \(ABCD\) cạnh \(a\). Gọi \(M,N,P,I,J,K\)lần lượt là trung điểm của các cạnh \(AB,BC,CD,AC,AD,DB\).
Ta có: \(IM = IN = NM = \frac{1}{2}a\) (tính chất đường trung bình của tam giác). Suy ra \(IMN\) đều.
Chứng minh tương tự, ta có các tam giác: \(IPN\), \(IPJ\), \(KPJ\), \(KPN\), \(IMJ\), \(KMJ\), \(KMN\) là các tam giác đều.
Tám tam giác trên tạo thành một đa diện có các đỉnh là \(M,N,P,I,J,K\)mà mỗi đỉnh là đỉnh chung của đúng \(4\)tam giác đều. Do đó đa diện đó là đa diện đều loại \(\left\{ {3;4} \right\}\) tức là bát diện đều.
Gọi\(P,I,J,K\) là tâm của các mặt \(ABD\), \(ACD\), \(ABC\), \(BCD\) của tứ diện đều \(ABCD\).
Ta có: \(\frac{{IN}}{{AN}} = \frac{{KN}}{{BN}} = \frac{1}{3} \Rightarrow \frac{{KI}}{{BA}} = \frac{1}{3} \Rightarrow KI = \frac{1}{3}a\).
Chứng mình tương tự ta có: \(IK = JP = IJ = PI = PK = KI = \frac{1}{3}a\).
Vậy \(PIJK\) là tứ diện đều.
Theo SGK Hình học \(12\) trang \(17\) thì khối đa diện đều loại \(\left\{ {3\,;\,5} \right\}\) là khối hai mươi mặt đều.
Hình \(1\), Hình \(2\), Hình \(4\) không phải hình đa diện vì nó vi phạm tính chất: “ mỗi cạnh là cạnh chung của đúng hai mặt”.
Đó là các mặt phẳng \(\left( {SAC} \right)\), \(\left( {SBD} \right)\), \(\left( {SHJ} \right)\), \(\,\left( {SGI} \right)\) với \(G\), \(H\), \(I\), \(J\) là các trung điểm của các cạnh \(AB,\)\(CB,\)\(CD,\)\(AD\) (hình vẽ bên dưới).
Ba khối tứ diện là \(AA’B’C’\), \(ABB’C’\), \(ABCC’\).
Ta có \[{S_{ABCD}} = {a^2}\]. \[{V_{S.ABCD}} = \frac{1}{3}SA.{S_{ABC{\rm{D}}}} = \frac{{\sqrt 2 {a^3}}}{3}\].
\(\frac{a\sqrt{3}}{2}\)
\( a \sqrt{3}\).
\\
\({V_{S.ABC}} = \frac{1}{3}.{S_{\Delta ABC}}.SA \Rightarrow SA = \frac{{3{V_{S.ABC}}}}{{{S_{\Delta ABC}}}} = \frac{{3.\frac{{{a^3}}}{4}}}{{\frac{{{a^2}\sqrt 3 }}{4}}} = a\sqrt 3 \).
\(V_{S \cdot A B C}=\frac{1}{3} \cdot S_{\triangle A B C} \cdot S A \Rightarrow S A=\frac{3 V_{S \cdot A B C}}{S_{\triangle A B C}}=\frac{3 \cdot \frac{a^3}{4}}{\frac{a^2 \sqrt{3}}{4}}=a \sqrt{3}\)
Ta có: \(S_{A B C D}=a^2\).
Gọi \(H\) là tâm của hình vuông \(A B C D\). Tam giác \(A S C\) là tam giác vuông, \(H\) là trung điểm của \(A C\) nên \(S H=\frac{A C}{2}=\frac{a \sqrt{2}}{2}\).
Vậy \(V_{S \cdot A B C D}=\frac{1}{3} S_{A B C D} \cdot S H=\frac{1}{3} \cdot a^2 \cdot \frac{a \sqrt{2}}{2}=\frac{a^3 \sqrt{2}}{6}\).
Ta có, với \(x>0: P=\sqrt[4]{x \cdot \sqrt[3]{x^2 \cdot \sqrt{x^3}}}=\sqrt[4]{x \cdot \sqrt[3]{x^2 \cdot x^{\frac{3}{2}}}}=\sqrt[4]{x \cdot \sqrt[3]{x^{\frac{7}{2}}}}=\sqrt[4]{x \cdot x^{\frac{7}{6}}}=\sqrt[4]{x^{\frac{13}{6}}}=x^{\frac{13}{24}}\).
Ta có.
$$
\begin{aligned}
& a^{\frac{1}{2}}>a^{\frac{1}{3}} \Leftrightarrow \frac{1}{2} \ln a>\frac{1}{3} \ln a \Leftrightarrow \frac{1}{6} \ln a>0 \Leftrightarrow a>1 \\
& b^{\frac{2}{3}}>b^{\frac{3}{4}} \Leftrightarrow \frac{2}{3} \ln b>\frac{3}{4} \ln b \Leftrightarrow 0>\frac{1}{12} \ln b \Leftrightarrow 0<b<1
\end{aligned}
$$
Ta có.
$$
\begin{aligned}
& a^{\frac{1}{2}}>a^{\frac{1}{3}} \Leftrightarrow \frac{1}{2} \ln a>\frac{1}{3} \ln a \Leftrightarrow \frac{1}{6} \ln a>0 \Leftrightarrow a>1 \\
& b^{\frac{2}{3}}>b^{\frac{3}{4}} \Leftrightarrow \frac{2}{3} \ln b>\frac{3}{4} \ln b \Leftrightarrow 0>\frac{1}{12} \ln b \Leftrightarrow 0<b<1
\end{aligned}
$$
\(\underline{L ư u} ý:\) Ta có thể sử dụng máy tính Casio để thử các đáp án bằng cách cho \(a, b\) các giá trị cụ thể.
Tính đạo hàm của hàm số \(y = {\log _{2019}}\left| x \right|,\,\forall x \ne 0\).
Một học sinh \(A\) khi 15 tuổi được hưởng tài sản thừa kế \(200000000 VNĐ \). Số tiền này được bảo quản trong ngân hàng \(B\) với kì hạn thanh toán 1 năm và học sinh \(A\) chỉ nhận được số tiền này khi 18 tuổi. Biết rằng khi 18 tuổi, số tiền mà học sinh \(A\) được nhận sẽ là 231525000 VNĐ. Vậy lãi suất kì hạn một năm của ngân hàng \(B\) là bao nhiêu?
\end{document}
